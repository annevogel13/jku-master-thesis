% !TeX program = xelatex
% !TeX encoding = UTF-8
% !TeX spellcheck = en_US
% !BIB program = biber
%% 
%% The above lines help editors like TeXstudio to automatically choose the right tools
%% to compile your LaTeX source file. If your tool does not support these magic comments,
%% you will need to make appropriate manual choices.
%% 
%% You can safely use "pdflatex" instead of "xelatex" if you prefer the pdfLaTeX toolchain.
%% However, pdfLaTeX will not be able to deliver the professional font experience that you
%% will get with XeLaTeX. You can also safely use "lualatex" instead of "xelatex" while
%% preserving the professional font experience if you prefer the LuaLaTeX toolchain.
%% 
%% _Important_: These magic comments should be on the first lines of your source file.
%% 
%%%%%%%%%%%%%%%%%%%%%%%%%%%%%%%%%%%%%%%%%%%%%%%%%%%%%%%%%%%%%%%%%%%%%%%%%%%%%%%%

%%%%%%%%%%%%%%%%%%%%%%%%%%%%%%%%%%%%%%%%%%%%%%%%%%%%%%%%%%%%%%%%%%%%%%%%%%%%%%%%
%% 
%%            JJJJ   K                         K   UUUU         UUUU  
%%            JJJJ   KKKK                   KKKK   UUUU         UUUU  
%%            JJJJ   KKKKKK               KKKKKK   UUUU         UUUU  
%%            JJJJ      KKKKKK         KKKKKK      UUUU         UUUU  
%%            JJJJ         KKKKKK   KKKKKK         UUUU         UUUU  
%%            JJJJ            KKKKKKKKK            UUUU         UUUU  
%%    JJ     JJJJJ               KKK               UUUUU       UUUUU  
%%  JJJJJJJJJJJJJ    KKKKKKKKKKKKKKKKKKKKKKKKKKK    UUUUUUUUUUUUUUU   
%%    JJJJJJJJJ      KKKKKKKKKKKKKKKKKKKKKKKKKKK      UUUUUUUUUUU     
%% 
%% This is an example file for using the JKU LaTeX technical report template
%% for your thesis.
%% 
%% Template created by Michael Roland (2021)
%% 
%%%%%%%%%%%%%%%%%%%%%%%%%%%%%%%%%%%%%%%%%%%%%%%%%%%%%%%%%%%%%%%%%%%%%%%%%%%%%%%%

%%%%%%%%%%%%%%%%%%%%%%%%%%%%%%%%%%%%%%%%%%%%%%%%%%%%%%%%%%%%%%%%%%%%%%%%%%%%%%%%
%% 
%% Document class: This is a koma-script book.
%% 
\documentclass[a4paper,oneside,10pt,ngerman,english]{scrbook}
%% 
%% The comma-separated list in square brackets are class options.
%% Useful options that you might want to use:
%% 
%% Paper size:
%%  * a4paper ... A4 paper size
%% 
%% Optimize for single-sided or double-sided printing:
%%  * oneside ... single-sided
%%  * twoside ... double-sided
%% 
%% Base font size:
%%  * 10pt ... 10-pt font is used for normal text
%%  * 11pt ... 11-pt font is used for normal text
%% 
%% Define document languages (the last specified language becomes the document default
%% language):
%%  * ngerman ... German
%%  * english ... English
%%  * ...
%% 
%% Alternate document classes: While the JKU report template supports the koma-script classes
%% `scrartcl', `scrreprt' and `scrbook', your should always use the book class `scrbook' for
%% your thesis.
%%  
%% _Important_: The document class should be the first line of LaTeX code in your main
%% source file. Do not place anything but comments / magic comments above that line (unless
%% you really know what you are doing).
%% 
%%%%%%%%%%%%%%%%%%%%%%%%%%%%%%%%%%%%%%%%%%%%%%%%%%%%%%%%%%%%%%%%%%%%%%%%%%%%%%%%

%%%%%%%%%%%%%%%%%%%%%%%%%%%%%%%%%%%%%%%%%%%%%%%%%%%%%%%%%%%%%%%%%%%%%%%%%%%%%%%%
%% 
%% Treat input files as UTF-8 encoded. Make sure to always load this when you use pdfLaTeX
%% so that pdfLaTeX knows how to read and interpret characters in this source file.
%% 
\usepackage[utf8]{inputenc}
%% 
%%%%%%%%%%%%%%%%%%%%%%%%%%%%%%%%%%%%%%%%%%%%%%%%%%%%%%%%%%%%%%%%%%%%%%%%%%%%%%%%

%%%%%%%%%%%%%%%%%%%%%%%%%%%%%%%%%%%%%%%%%%%%%%%%%%%%%%%%%%%%%%%%%%%%%%%%%%%%%%%%
%% 
%% Use the JKU LaTeX technical report template for this document.
%% 
\usepackage[phdthesis,fancyfonts]{jkureport}
%% 
%% The comma-separated list in square brackets are theme options. Useful options that you
%% might want to use:
%% 
%% Document type:
%%  * phdthesis     ... PhD thesis.
%%  * mathesis      ... Master's thesis.
%%  * diplomathesis ... Diploma thesis.
%%  * bathesis      ... Bachelor's thesis.
%%  * seminarreport ... Seminar report.
%%  * techreport    ... Technical report.
%% 
%% Color scheme selection options:
%%  * JKU  ... Use JKU (gray) color scheme (this is the default if no scheme is selected).
%%  * BUS  ... Use Business School color scheme.
%%  * LIT  ... Use Linz Institute of Technology color scheme.
%%  * MED  ... Use MED faculty color scheme.
%%  * RE   ... Use RE faculty color scheme.
%%  * SOE  ... Use School of Education color scheme.
%%  * SOWI ... Use SOWI faculty color scheme.
%%  * TNF  ... Use TNF faculty color scheme.
%% 
%% Space-efficient monospace font options (requires XeTeX/LuaTeX):
%%  * compactmono   ... Use condensed fixed-width font everywhere.
%%  * nocompactverb ... Do not use condensed fixed-width font for verbatim and listings.
%% 
%% Style-breaking options:
%%  * noimprint      ... Do not insert imprint on title pages.
%%  * nojkulogo      ... Do not insert JKU & K logos on title pages.
%%  * capstitle      ... Set document title in capital letters.
%%  * nofancyfonts   ... Do not use custom TTF fonts with XeTeX/LuaTeX / supress pdfLaTeX warning.
%%  * equalmargins   ... Decrease the outer page margin to have both page margins of equal size
%%                       (the additional outer margin is intentional and to be used for
%%                       anotations; equalmargins also causes the text width to be
%%                       significantly larger than optimal for reading).
%% 
%% Experimental options:
%%  * mathastext ... Use standard document fonts (enhanced with symbols from Fira Math font
%%                   when using XeTeX/LuaTeX) in math mode.
%% 
%% Advanced options:
%%  * legacymode={<mode>} ... Activate a legacy mode to account for previous formal requirements,
%%                            styles, and/or features. May be specified mutliple times to activate
%%                            multiple legacy modes. The following <mode>s are supported:
%%      - phd2021             ... Adapt cover sheet to formal requirements for unstructured
%%                                doctoral programs before 2021W (also accounts for change from
%%                                "Beurteiler*in" to "Betreuer*in" in 2024W versions of the German
%%                                thesis cover sheets)
%%  * noautopdfinfo       ... Do not automatically try to add pdfinfo with hyperref from document
%%                            metadata fields.
%%  * logopath={<path>}   ... Set the path where the theme can find its own logo resources. This
%%                            should typically be a relative path and the default is `./logos'.
%%  * fontpath={<path>}   ... Set the path where the theme can find its own font resources. This
%%                            should typically be a relative path and the default is `./fonts'.
%% 
%% Hint: Boolean options can be used in the forms `option' or `option=true' the enable the
%% option and `nooption' or `option=false' to disable the option.
%% 
%%%%%%%%%%%%%%%%%%%%%%%%%%%%%%%%%%%%%%%%%%%%%%%%%%%%%%%%%%%%%%%%%%%%%%%%%%%%%%%%

%%%%%%%%%%%%%%%%%%%%%%%%%%%%%%%%%%%%%%%%%%%%%%%%%%%%%%%%%%%%%%%%%%%%%%%%%%%%%%%%
%% 
%% This is the place where you can load additional packages. If you want to load
%% a package `biblatex', you would use the command `\usepackage{biblatex}'.
%% 

\usepackage{csquotes}
\usepackage[backend=biber,citestyle=numeric,sortcites=true,maxcitenames=2,style=ACM-Reference-Format]{biblatex}
\setcounter{biburlnumpenalty}{100} %% reducing biburl* penalties typically improves URL placement in bibliography
\setcounter{biburllcpenalty}{100}
\setcounter{biburlucpenalty}{100}
\usepackage{todonotes}
\usepackage{import}
\usepackage{amsfonts}
\usepackage{subfigure}
%\usepackage{acronym}

%% 
%%%%%%%%%%%%%%%%%%%%%%%%%%%%%%%%%%%%%%%%%%%%%%%%%%%%%%%%%%%%%%%%%%%%%%%%%%%%%%%%

%%%%%%%%%%%%%%%%%%%%%%%%%%%%%%%%%%%%%%%%%%%%%%%%%%%%%%%%%%%%%%%%%%%%%%%%%%%%%%%%
%% 
%% Bibliography data files.
%% 

\addbibresource{references.bib}

%% 
%%%%%%%%%%%%%%%%%%%%%%%%%%%%%%%%%%%%%%%%%%%%%%%%%%%%%%%%%%%%%%%%%%%%%%%%%%%%%%%%

\begin{document}
%% Begin with the frontmatter (among other things, this switches to roman page numbers)
\frontmatter

%%%%%%%%%%%%%%%%%%%%%%%%%%%%%%%%%%%%%%%%%%%%%%%%%%%%%%%%%%%%%%%%%%%%%%%%%%%%%%%%
%% 
%% Thesis information and title page
%% 

%% Command \title{title}: sets the title of your thesis
\title{Space for your thesis title}

%% Command \titleshort{short title}: sets an abbreviated version of the thesis title for page heads
%\titleshort{Optional space for your abbreviated title}

%% Command \subtitle{subtitle}: sets the subtitle for seminar/technical reports (not used for theses)
%\subtitle{Seminar Report}

%% Command \author{name}: sets the author's name; use \prefix{} and \suffix{} to add academic titles and suffixes, use \matno{} to add the immatriculation number
\author{\prefix{DI} Firstname~Lastname \suffix{BSc}\matno{12345678}}

%% Command \supervisor[number,gender]{name}: sets the name of the supervisor (where number optionally
%%   defines the rank of the supervisor (1-3) and gender specifies if the supervisor is male or female
%%   to adapt gender-specific terms in German)
\supervisor{\prefix{Prof. Dr.} Firstname~Lastname}
\supervisor[2]{\prefix{Prof. Dr.} Firstname~Lastname}

% Command \assistantsupervisor{name}: sets the name of the assistant supervisor(s)
\assistantsupervisor{\prefix{Dr.} Firstname~Lastname \suffix{MSc}}

% Command \degree{degree}{degree program}: sets the degree and degree program name
\degree{Master of Science}{Computer Science}

% Command \submissiondepartment{institute or department}: set the department that the thesis is submitted at
\submissiondepartment{Institute of Networks~and~Security}

% Command \submissionplace{place}: set the place of submission (typically `Linz', which is also the default)
%\submissionplace{Linz}

% Command \date{YYYY-MM-DD}: set the day of submission (defaults to today)
%\date{2020-04-09}

% Command \keywords{text}: set the document keywords
%\keywords{Space for your comma-separated keywords}


%% Finally, print the title page using the above information:
\maketitle
%% 
%%%%%%%%%%%%%%%%%%%%%%%%%%%%%%%%%%%%%%%%%%%%%%%%%%%%%%%%%%%%%%%%%%%%%%%%%%%%%%%%

%%%%%%%%%%%%%%%%%%%%%%%%%%%%%%%%%%%%%%%%%%%%%%%%%%%%%%%%%%%%%%%%%%%%%%%%%%%%%%%%
%% 
%% Include your abstract into the frontmatter
%% 

\import{./}{00-abstract}

%% 
%%%%%%%%%%%%%%%%%%%%%%%%%%%%%%%%%%%%%%%%%%%%%%%%%%%%%%%%%%%%%%%%%%%%%%%%%%%%%%%%

%%%%%%%%%%%%%%%%%%%%%%%%%%%%%%%%%%%%%%%%%%%%%%%%%%%%%%%%%%%%%%%%%%%%%%%%%%%%%%%%
%% 
%% Include your acknowledgements into the frontmatter (optional and typically only
%% used to acknowledge external funding, discuss with your supvervisor if unsure)
%% 

%\import{./}{acknowledgements}

%% 
%%%%%%%%%%%%%%%%%%%%%%%%%%%%%%%%%%%%%%%%%%%%%%%%%%%%%%%%%%%%%%%%%%%%%%%%%%%%%%%%

%%%%%%%%%%%%%%%%%%%%%%%%%%%%%%%%%%%%%%%%%%%%%%%%%%%%%%%%%%%%%%%%%%%%%%%%%%%%%%%%
%% 
%% Add a table of contents (and optionally various other lists)
%% 

%% Make sure to start the table of contents on a new odd page (odd is only relevant in twoside layout)
\cleardoubleoddpage
%% Print the table of contents
\tableofcontents

%% Make sure to start the list of tables on a new odd page (odd is only relevant in twoside layout)
%\cleardoubleoddpage
%% Print the list of tables (optional and often not necessary)
%\listoftables

%% Make sure to start the list of figures on a new odd page (odd is only relevant in twoside layout)
%\cleardoubleoddpage
%% Print the list of figures (optional and often not necessary)
%\listoffigures

%% Make sure to start the list of acronyms on a new odd page (odd is only relevant in twoside layout)
%\cleardoubleoddpage
%% Include list of acronyms (optional and often not necessary)
%\import{./}{acronyms}

%% 
%%%%%%%%%%%%%%%%%%%%%%%%%%%%%%%%%%%%%%%%%%%%%%%%%%%%%%%%%%%%%%%%%%%%%%%%%%%%%%%%

%% Begin with the mainmatter (among other things, this switches to arabic page numbers)
\mainmatter

%%%%%%%%%%%%%%%%%%%%%%%%%%%%%%%%%%%%%%%%%%%%%%%%%%%%%%%%%%%%%%%%%%%%%%%%%%%%%%%%
%% 
%% Include your chapters
%% 

\import{./}{01-introduction}
\import{./}{02-example}
% ...
\import{./}{09-conclusion}

%% 
%%%%%%%%%%%%%%%%%%%%%%%%%%%%%%%%%%%%%%%%%%%%%%%%%%%%%%%%%%%%%%%%%%%%%%%%%%%%%%%%

%%%%%%%%%%%%%%%%%%%%%%%%%%%%%%%%%%%%%%%%%%%%%%%%%%%%%%%%%%%%%%%%%%%%%%%%%%%%%%%%
%% 
%% Print the bibliography
%% 
%% Make sure to start the bibliography on a new odd page (odd is only relevant in twoside layout)
\cleardoubleoddpage
\printbibliography
%% 
%%%%%%%%%%%%%%%%%%%%%%%%%%%%%%%%%%%%%%%%%%%%%%%%%%%%%%%%%%%%%%%%%%%%%%%%%%%%%%%%

%% Begin with the appendix part (all further chapters will be appendices)
\appendix

%%%%%%%%%%%%%%%%%%%%%%%%%%%%%%%%%%%%%%%%%%%%%%%%%%%%%%%%%%%%%%%%%%%%%%%%%%%%%%%%
%% 
%% Include your appendix chapters
%% 

\import{./}{91-appendix}
% ...

%% 
%%%%%%%%%%%%%%%%%%%%%%%%%%%%%%%%%%%%%%%%%%%%%%%%%%%%%%%%%%%%%%%%%%%%%%%%%%%%%%%%

\cleardoubleoddpage

\end{document}
\endinput
